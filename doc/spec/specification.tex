\documentclass[a4paper,12pt]{article}
\usepackage{amssymb} % needed for math
\usepackage{amsmath} % needed for math
\usepackage[utf8]{inputenc} % this is needed for german umlauts
\usepackage[ngerman]{babel} % this is needed for german umlauts
\usepackage[T1]{fontenc}    % this is needed for correct output of umlauts in pdf
\usepackage[margin=2.5cm]{geometry} %layout
\usepackage{booktabs}
\usepackage{tabularx}
\usepackage{array}

\usepackage{ngerman, longtable}

% this is needed for forms and links within the text
\usepackage[linktocpage]{hyperref}
\hypersetup{
ngerman,
colorlinks,
breaklinks,
urlcolor=red,
linkcolor=red,
citecolor=red}

% The following is needed in order to make the code compatible
% with both latex/dvips and pdflatex.
\ifx\pdftexversion\undefined
\usepackage[dvips]{graphicx}
\else
\usepackage[pdftex]{graphicx}
\DeclareGraphicsRule{*}{mps}{*}{}
\fi


\newcommand{\authorName}{Fabian Reiber}
\newcommand{\projektName}{Extrahieren von Metadaten aus öffentlich zugänglichen Exploits zur Anreicherung existierender Bedrohungsinformationen in MISP}
\newcommand{\tags}{\authorName, HAW Hamburg, Informatik, Exploit, MISP, Bedrohungsinformationen}
\title{\projektName}
\author{\authorName}
\date{\today}

\hypersetup{
  pdfauthor   = {\authorName},
  pdfkeywords = {\tags},
  pdftitle    = {\projektName}
} 
 
\newcommand\addrow[2]{#1 &#2\\ }

\newcommand\addheading[2]{#1 &#2\\ \hline}
\newcommand\tabularhead{\begin{tabular}{lp{13cm}}
\hline
}

\newcommand\addmulrow[2]{ \begin{minipage}[t][][t]{2.5cm}#1\end{minipage}% 
   &\begin{minipage}[t][][t]{8cm}
    \begin{enumerate} #2   \end{enumerate}
    \end{minipage}\\ }

\newenvironment{usecase}{\tabularhead}
{\hline\end{tabular}}


\begin{document}
\pagenumbering{roman}
\maketitle
\thispagestyle{empty} % no page number       
\pagebreak
\setcounter{page}{2}
\tableofcontents       
\clearpage
\pagenumbering{arabic}
 
\section{Einleitung}
In dieser Spezifikation wird ein System vorgestellt welches zum einen öffentlich zugängliche Exploits parsen und verschiedene Metadaten daraus gewinnen soll. Zum anderen sollen diese Informationen genutzt werden, um Bedrohungsinformationen aus MISP\footnote{https://www.misp-project.org/; zuletzt besucht am 01.12.2018} anzureichern. Das Dokument beinhaltet eine kurze Darstellung der Projektziele, Schnittstellen des Systems sowie der funktionalen und nichtfunktionalen Anforderungen. Darüber hinaus wird die Software-Architektur kurz vorgestellt.


\section{Konzept und Rahmenbedingungen}

\subsection{Zielsetzung}
\label{zeilsetzung}
Das zu entwickelnde System soll folgende Fragestellung konkret beleuchten: 'Welche Metadaten können aus vorhandenen Exploitinformationen extrahiert werden, um bestehende Bedrohungsinformationen in MISP sinnvoll anreichern zu können?'. Sinnvoll bedeutet hier zum einen, ob die extrahierten Daten den MISP-Analystinnen und Analysten erkenntnisreiche Informationen bereitstellen. Und zum anderen, ob diese Informationen genutzt werden können, um Exploits im Netzwerkdatenverkehr automatisiert erkennen zu können. Letzteres soll eine weitere Fragestellung für ein zusätzliches System darstellen.

\subsection{Schnittstellen}
Für die Umsetzung der beschriebenen Zielsetzung bedarf es Schnittstellen die das System nutzt. Die Exploits werden aus dem öffentlichen GitHub Repository von ExploitDB\footnote{https://github.com/offensive-security/exploitdb; zuletzt besucht am 01.12.2018} bezogen. Weiterhin ist eine Schnittstelle zu einer MISP Instanz notwendig, um aus ihr existierende Bedrohungsinformationen zu importieren und angereichert zurück zu exportieren.

\subsection{Meilensteine}
In der folgenden Tabelle wird eine erste Abschätzung von notwendigen Meilensteinen vorgestellt.
\begin{center}
\begin{tabular}{|c|c|c|}
\hline
\textbf{Nummer} & \textbf{Bezeichnung} & \textbf{Enddatum} \\
\hline
M1 & Erstellung des Konzepts & 13.12.2018 \\
\hline
M2 & Erstellung und Dokumentation der Architektur & 13.12.2018 \\
\hline
M3 & Erstellung von Statistiken & 31.12.2018 \\
\hline
M4 & Entwicklung der Exploit-Komponente & 31.01.2019 \\
\hline
M5 & Dokumentation der Exploit-Komponente & 31.01.2019 \\
\hline
M6 & Entwicklung der Anbindung an MISP & 28.02.2019 \\
\hline
M7 & Dokumentation der Anbindung an MISP & 28.02.2019 \\
\hline
\end{tabular}
\end{center}

\section{Funktionale Anforderungen}
\label{funk_anf}

\begin{longtable}[c]{|c|>{\centering\arraybackslash}p{5cm}|>{\centering\arraybackslash}p{6cm}|}
\hline \textbf{Nummer} & \textbf{Bezeichnung} & \textbf{Beschreibung}
\\\endhead
\hline 
FA01 & Herunterladen von Exploits & Das System soll die Exploits aus dem GitHub Repository herunterladen und auf der Festplatte speichern. \\
\hline
FA02 & Parsen und Speichern der Exploit Metadaten & Das System soll die vorhandenen Metadaten die in einer CSV-Datei vorliegen parsen und in eine Datenbank speichern. \\
\hline
FA03 & Erstellen von Statistiken & Das System soll visuelle Statistiken (Aggregationen, Zeitreihen, etc.) erzeugen und als Bilder auf der Festplatte speichern. \\
\hline
FA04 & Exploits Parsen & Das System soll die vorhandenen Exploits parsen. Das heißt der konkrete Exploitcode muss von den Kommentaren in der Exploitdatei unterschieden werden. \\
\hline
FA05 & Extrahieren von weiteren Metadaten & Das System soll aus den Kommentaren die in \textit{FA04} extrahiert wurden, weitere Metadaten generieren und die Daten in der Datenbank damit anreichern. \\
\hline
FA06 & Analyse des Exploitcodes & Das System soll eine Codeanalyse auf den Exploitcodes aus \textit{FA04} durchführen und die Daten in der Datenbank mit den gewonnenen Informationen anreichern. \\
\hline
FA07 & MISP Integration & Das System soll Bedrohungsinformationen aus MISP mit den gewonnenen Informationen aus den Exploits anreichern. \\
\hline
\end{longtable}

\section{Nichtfunktionale Anforderungen}
\begin{longtable}[c]{|c|>{\centering\arraybackslash}p{3cm}|>{\centering\arraybackslash}p{8cm}|}
\hline \textbf{Nummer} & \textbf{Bezeichnung} & \textbf{Beschreibung}
\\\endhead
\hline 
NF01 & Änderbarkeit & Das System soll modular aufgebaut werden, damit der Code wartbar ist und weitere Komponenten problemlos angegliedert werden können. \\
\hline
NF02 & Zeitverhalten & Das System soll in allen Bereichen der Datenverarbeitung (Parsen, Speichern, Codeanalyse, etc.) effizient arbeiten, damit Bedrohungsinformationen zügig mit weiteren Informationen angereichert werden können. \\
\hline
NF03 & Korrektheit & Das System soll korrekte Metadaten aus der Codeanalyse erzeugen. \\
\hline
NF04 & Fehlertoleranz & Das System soll bei der Anreicherung der Daten insofern fehlertolerant sein, sodass einzelne Datensätze unterschiedliche Attribute aufweisen können. \\
\hline
NF05 & Prüfbarkeit & Die Funktionsweisen einzelner Komponenten des Systems sollen durch Tests überprüfbar sein. \\
\hline
\end{longtable}

\section{Systemarchitektur}
Die in Abbildung \ref{img:arch} dargestellte Architektur spiegelt den ersten Entwurf des Systems wider. Die konkrete Fragestellung aus \ref{zeilsetzung} soll durch die Komponente \textit{Exploit Information Retrieval} repräsentiert werden. Wohingegen die MISP-Komponente eine existierende MISP Instanz zeigen und die \textit{Exploit Detection}-Komponente die zweite Fragestellung aus der Zielsetzung visualisieren soll. Die \textit{Exploit Execution}-Komponente zeigt den möglichen Fall, dass die Exploits ggf. in einer Sandbox zur Ausführung gebracht werden müssen, um weitere Metadaten erhalten zu können. 

Die funktionalen Anforderungen aus \ref{funk_anf} lassen sich in der \textit{Exploit Information Retrieval}-Komponente widerfinden. Der \textit{Exploit Fetcher} soll für das Herunterladen der Exploits zuständig sein und soll diese lokal auf der Festplatte abspeichern (\textit{FA01}). Der \textit{CSV Parser} soll die Metadaten parsen und in einer Datenbank ablegen (\textit{FA02}). Durch das \textit{Aggregation}-Modul sollen die Statistiken erstellt werden (\textit{FA03}). Der \textit{Exploit Parser} soll auf den lokalen Exploit Dateien zugreifen und diese parsen können (\textit{FA04}). Diese Informationen sollen sodann an das Modul \textit{Data Augmentation} weitergeleitet werden, welches zum einen weitere Metadaten extrahieren (\textit{FA05}) und zum anderen eine Codeanalyse der Exploits durchführen soll (\textit{FA06}). Die Informationen in der Datenbank sollen durch die ermittelten Informationen aus der Extraktion und Analyse angereichert werden. Zuletzt soll der \textit{MISP Integrator} potentielle Bedrohungsinformationen ermitteln, mit den gewonnenen Informationen anreichern und wieder zurück exportieren (\textit{FA07}).

\begin{figure}[h]
\centering
\includegraphics[scale=0.55]{../arch/arch.pdf}
\caption{Entwurf einer Systemarchitektur}
\label{img:arch}
\end{figure}

\clearpage
%\input{glossary} 
\end{document}
